% -*- mode: latex; mode: reftex; mode: auto-fill; mode: flyspell; coding: utf-8; tex-command: "pdflatex.sh" -*-

\documentclass[oneside,11pt]{memoir}

\input{layout.tex}

\usepackage{lipsum} 

\hypersetup{
  pdfauthor={Noor Chauhan},
  pdftitle={The Little Book of Python},
  pdfsubject={Python},
  pdfkeywords={Python},
  pdfproducer={LaTeX and TikZ},
  pdfcreator={pdflatex},
}

%%%%%%%%%%%%%%%%%%%%%%%%%%%%%%%%%%%%%%%%%%%%%%%%%%%%%%%%%%%%%%%%%%%%%%%%%%%%%%%%%%%%%%%%%%%

\begin{document}

\thispagestyle{empty}

\begin{center}

\vspace*{\stretch{1}}

{\huge The Little Book\\[0.75ex] of\\[1.75ex] Python}
\vspace*{4ex}


\vspace*{\stretch{1}}

\end{center}

\newpage

%%%%%%%%%%%%%%%%%%%%%%%%%%%%%%%%%%%%%%%%%%%%%%%%%%%%%%%%%%%%%%%%%%%%%%%%%%%%%%%%%%%%%%%%%%%

\vspace*{\stretch{1.25}}

This ebook is formatted to fit on a phone screen.

\vspace*{\stretch{1}}

\begin{flushright}
\footnotesize beta-\dotdate\today
\end{flushright}

\vspace*{-3ex}

\newpage

%%%%%%%%%%%%%%%%%%%%%%%%%%%%%%%%%%%%%%%%%%%%%%%%%%%%%%%%%%%%%%%%%%%%%%%%%%%%%%%%%%%%%%%%%%%
% Table of content
%%%%%%%%%%%%%%%%%%%%%%%%%%%%%%%%%%%%%%%%%%%%%%%%%%%%%%%%%%%%%%%%%%%%%%%%%%%%%%%%%%%%%%%%%%%

{
\everymath{\color{black}}
\tableofcontents* % Prints the table of contents
%\addcontentsline{toc}{chapter}{Contents}
}

\clearpage

\listoffigures*
\addcontentsline{toc}{chapter}{List of figures}

%%%%%%%%%%%%%%%%%%%%%%%%%%%%%%%%%%%%%%%%%%%%%%%%%%%%%%%%%%%%%%%%%%%%%%%%%%%%%%%%%%%%%%%%%%%

\chapter*{Foreword}
\addcontentsline{toc}{chapter}{Foreword}
This book is inspired by The Little Book of Deep Learning by prof. Francois Fleuret \cite{} and aims to develop a guide on python with the focus on machine learning and deep learning fundamental techniques with PyTorch or python in general. This book is currently a collective effort, open-source and welcomes contribution from authors and programmers.

%%%%%%%%%%%%%%%%%%%%%%%%%%%%%%%%%%%%%%%%%%%%%%%%%%%%%%%%%%%%%%%%%%%%%%%%%%%%%%%%%%%%%%%%%%%

\part{History}


%% This first part provides a minimal background about machine
%% learning, issues and techniques for efficient computation, and the
%% strategies to train a parametric model.

%%%%%%%%%%%%%%%%%%%%%%%%%%%%%%%%%%%%%%%%%%%
\chapter{Introduction}
Guido van Rossum born 31 January 1956 is a Dutch programmer and the creator of the Python programming language. Python programming language is comparatively underrated as it supports programming paradigms such as structured, object-oriented and functional. It has support from multiple organizations writing and maintaining the collectively large standard and user function libraries.



%%%%%%%%%%%%%%%%%%%%%%%%%%%%%%%%%%%%%%%%%%%
\chapter{Deep learning fundamentals}

%%%%%%%%%%%%%%%%%%%%%%%%%%%%%%%%%%%%%%%%%%%%%%%%%%%%%%%%%%%%%%%%%%%%%%

%%%%%%%%%%%%%%%%%%%%%%%%%%%%%%%%%%%%%%%%%%%%%%%%%%%%%%%%%%%%%%%%%%%%%%%%%%%%%%%%%%%%%%%%%%%

\bibliography{test}

%% If there is an index, it should follow the bibliography (see the
%% Manual, 14.62). Bibliography entries are listed in one alphabetical
%% sequence arranged by the surname of the first author or by title if
%% there is no author.25 Nov 2022
%%
%% Bibliography - Chicago Citation Style, 17th Edition - Library

\printindex

%%%%%%%%%%%%%%%%%%%%%%%%%%%%%%%%%%%%%%%%%%%%%%%%%%%%%%%%%%%%%%%%%%%%%%%%%%%%%%%%%%%%%%%%%%%

\newpage

\vspace*{\stretch{1}}

\ifdefined\draft
\begin{center}
  {\color{red} (draft, do not circulate)}
\end{center}
\else
This book is licensed under the
\href{https://creativecommons.org/licenses/by-nc-sa/4.0/}{Creative
  Commons BY-NC-SA 4.0 International License.}
\fi

Template by \href{https://fleuret.org/francois/}{François Fleuret}.

\vspace*{\stretch{1}}

%%%%%%%%%%%%%%%%%%%%%%%%%%%%%%%%%%%%%%%%%%%%%%%%%%%%%%%%%%%%%%%%%%%%%%%%%%%%%%%%%%%%%%%%%%%

\checknbdrafts

\end{document}
